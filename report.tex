\documentclass[a4paper,11pt]{report}
\usepackage[utf8x]{inputenc}
\usepackage{amsmath, amssymb, amsthm} % AMS packages
\usepackage{graphicx,color}           % Packages for graphics and color
\usepackage[left=1.5in, right=1in, top=1in, bottom=1in, includefoot, headheight=13.6pt]{geometry}
\usepackage[T1]{fontenc}               % Ensure correct font encoding
\usepackage{hyperref}
\usepackage{algorithmic}
\usepackage{algorithm}
\usepackage{verbatim}

\renewcommand{\chaptername}{}
\renewcommand{\thechapter}{}
\renewcommand{\thesection}{}


\begin{document}
\title{
\huge{\textbf{Training Neural Networks for Classification using a Gene Expression Programming Framework}}\\[1.4cm]
\large{by}\\[0.2cm]
\large{Morley Abbott, 100744273} \\[1.4cm]
\large{Supervised by Dr. Anthony White} \\[0.2cm]
\large{School of Computer Science}\\[1.4cm]
\large{Honours Project (COMP 4905)} \\[0.2cm]
\large{Submitted in partial fullfilment of the} \\[0.2cm]
\large{requirements for the degree of } \\[0.2cm]
\large{Bachelor of Computer Science} \\[1.4cm]
\large{at}\\[1.4cm]
\large{Carleton University} \\[0.2cm]
\large{Ottawa, Ontario, Canada} \\[0.2cm]
\large{April, 2011} \\[0.2cm]}
\author{} \date{}

\maketitle


\chapter*{Abstract}

Classification is a major part of machine learning, and is used everyday on large scales:
Google identifies different types of searches and users; Amazon identifies
what kind of people will like a given book; medical clinics can quickly and accurately 
give a diagnosis for a list of symptoms. The basic pattern of converting an input vector 
of data into one item from a set of possible classes is used in applications everywhere. 
Much research is done towards finding more powerful tools to accurately build these classification 
systems.

Evolutionary Computing a powerful and useful tool in many application 
areas in which the search space of potential solutions is very large, and good solutions
can relate to one another. By implementing nature's evolution algorithms in our own ways, we can quickly 
find partial solutions, and combine them into better solutions. This simple process can be 
used in a variety of applications. One such area is artificial neural networks, used to model complex
relationships between inputs and outputs, and discover patterns in data.

This paper utilizes a specific type of evolutionary algorithm, Gene Expression Programming,
and uses it to create artificial neural networks for classification. To do this, a highly expandable 
framework for using Gene Expression Programming is presented, along with some results of tests done 
with it. The paper will begin with a description of the Gene Expression Programming algorithm, and 
how it can be used to create artificial neural networks. This will be followed by an explanation 
of the framework developed, how it works, and some experiments done using the framework. Finally,
the results of the tests will be presented, along with some conclusions, a description of the 
concepts learned, and a short discussion on potential future work with the framework. 


\addcontentsline{toc}{chapter}{Acknowledgements}
\chapter*{Acknowledgements}



\tableofcontents 

\addcontentsline{toc}{chapter}{List of Figures}
\listoffigures

\addcontentsline{toc}{chapter}{List of Tables}
\listoftables

\chapter{Terminology}
-GEP : Gene Expression Programming
-GEA : Gene Expression Algorithm


\chapter{Introduction}

\section{Motivation}

Classification and pattern recongition are powerful tools which can be used in a variety of 
fields. Online businesses can best decide what ads to show which users in order to maximize
the likelihood of the ad being noticed and clicked on. Medical systems can help doctors 
decide if a patient is well enough to be sent home. Meteorologists can determine what the 
most likely weather in a given city will be tomorrow. 

All of these tasks require that a given 
training set of data be used to create the most accurate classification system possible. 
This has implications in information theory: how much entropy is there in the data, and 
how high can classification possibly be for a given training set? Research in this field 
always comes down to how accurate the generated classifiers can be. 

The motivation of this project is to create a framework for a specific type of classifier
generation, Gene Expression Programming, to see just how powerful the technique is, and 
allow others to research further using it.

\section{Objectives}

The objectives of the project come down to a set of goals for the final product:

\begin{enumerate}
 \item Build a framework which implements the Gene Expression Programming Algorithm.
 \item Make the framework expandable, so that future users can enhance it further, and use it for their purposes. 
 \item Use the framework to see how powerful classifiers built with it can be, with just the basic tools created
alongside the framework. 
\end{enumerate}

\chapter{Background}

\section{Biological Inspirations}

Nearly every major part of this project takes its inspiration from biology. As such, the biological inspirations 
should first be explained. 

%the difference between a genotype and a phenotype
The first important concept is that of genotypes and phenotypes. In nature, the genotype is the genetic makeup 
of a cell, and an individual. For most of life as we know it, this comes down to DNA, our basic building block 
of life. However, DNA on its own is of little use: it is primarily a storage of information. 
Instead, our DNA gets expressed as RNA, and proteins, which do the real work of building living creatures. 

%how this is useful in nature
This separation of genotype and phenotype is useful to nature. It allows for genetic operators, like 
mutation, and genetic recombination, to play their 
role on the genotype alone, where the information is stored, rather than on the expressed phenotype, which 
has purpose and complex structure. This makes these genetic operators much easier, as complex structures 
are usually not very adept at modification. The genotype however, like our DNA, is as simple as a string 
of letters. 

%Neural Networks in nature
The other piece of nature that the project has drawn inspiration from is Neural Networks. Inside a brain, 
be it that of a human being or a fruit fly, is a network of neuron cells. These neurons fire electrical 
signals between their neighbours, strengthening and altering their connections to one another. The end 
result of all this is control over the rest of the body and, especially in the case of humans, learning
of patterns in the surrounding world. 

\section{Genetic Algothims}

%how genetic algorithms work in nature, the importance of recombination
Nature's genetic algorithm is often sumamrized as 'survival of the fittest'. In each successive 
generation, some have traits which make them more successful than others. From that success,
they manage to create more offspring which retain their traits, which gives them similar advantages. 
The other important piece of this survival is the combination of traits of two parents into a child
with, potentially, many of the advantages of both. If one parent is faster, and one parent is stronger, 
the child may be both fast and strong. This can give it the advantages of both 
parents, meaning it will likely have more offspring than average. 
It may also be the case that the offspring gets neither of the 
traits from its parents- in which case, it may not be fit enough to have any offspring at all.

Repeating this process over many generations results in the most fit individuals having their traits 
passed onto a large part of the population. With this, nature also adds occasional random mutation, 
which has many potential sources, which can improve an individual by chance, in ways that did not 
previously exist in the population.

%how this method can be applied to CS
This Genetic Algorithm can be applied to our purposes. For a given search space of possible solutions, 
we can optimize variables to get the result we want most. We can choose the fitness function, and the selection 
function, as well as how we wish to combine and mutate individuals in the population. 

The cannonical genetic algorithm is presented:

\begin{algorithm}
\caption{Algorithm 1: Genetic Algorithm}
\renewcommand{\algorithmicrequire}{\textbf{Input:}}
\renewcommand{\algorithmicensure}{\textbf{Output:}}
\begin{algorithmic}[1]
\REQUIRE A problem, $\rho$
\ENSURE A solution, or set of solutions, S
\STATE $P \leftarrow$ Generate-Initial-Population
\WHILE{ \textit{not} termination-conditions-met } 
  \STATE $F \leftarrow $ computeFitness($P$)
  \STATE rankAndSelect($P, F$)
  \STATE combinePopulation($P$)
  \STATE mutatePopulation($P$)
\ENDWHILE

\end{algorithmic}
\end{algorithm}  



%quick explanation of the abstract genetic algorithm, 

\section{Gene Expression Programming}

%What a parse tree is

%how a parse tree can be defined by a string, given bla

%how the GEA uses this to take a karva string, and make an expressed phenotype

%how evolution plays a role

%RNCs, and their role (reference Ferreira)


\section{Artifical Neural Networks}

%Description of a single perceptron/neuron

%description of a layer of neurons

%Description of a multi-layer perceptron, or ANN

%hidden layers, why they matter

%activator functions, how they work classically.


\section{Applying Gene Expression Programming to Artificial Neural Networks}

%explanation of main concept: replace the activator functions with expressed karva strings

%why this is so powerful (can do anything classic can do plus more, disregard useless data, find unusual relationships)


\chapter{The GEP Framework}

%High level view of the framework- it applies the algorithm


%customizability of functions, fitness, selection


%pipeline framework: add whatever you like


%examples of pipelined features, current and potential future


\chapter{Experiments Conducted}



\chapter{Results}




\chapter{Conclusion}


\chapter{References}

%FIND SOURCE:
%-Genetypes vs Phenotypes



\end{document}          
