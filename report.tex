\documentclass[a4paper,11pt]{report}
\usepackage[utf8x]{inputenc}
\usepackage{amsmath, amssymb, amsthm} % AMS packages
\usepackage{graphicx,color}           % Packages for graphics and color
\usepackage[left=1.5in, right=1in, top=1in, bottom=1in, includefoot, headheight=13.6pt]{geometry}
\usepackage[T1]{fontenc}               % Ensure correct font encoding
\usepackage{hyperref}


\renewcommand{\chaptername}{}
\renewcommand{\thechapter}{}
\renewcommand{\thesection}{}


\begin{document}
\title{
\huge{\textbf{Training Neural Networks for Classification using a Gene Expression Programming Framework}}\\[1.4cm]
\large{by}\\[0.2cm]
\large{Morley Abbott, 100744273} \\[1.4cm]
\large{Supervised by Dr. Anthony White} \\[0.2cm]
\large{School of Computer Science}\\[1.4cm]
\large{Honours Project (COMP 4905)} \\[0.2cm]
\large{Submitted in partial fullfilment of the} \\[0.2cm]
\large{requirements for the degree of } \\[0.2cm]
\large{Bachelor of Computer Science} \\[1.4cm]
\large{at}\\[1.4cm]
\large{Carleton University} \\[0.2cm]
\large{Ottawa, Ontario, Canada} \\[0.2cm]
\large{April, 2011} \\[0.2cm]}
\author{} \date{}

\maketitle


\chapter*{Abstract}

Classification is a major part of machine learning, and is used everyday on large scales:
Google identifies different types of searches and users; Amazon identifies
what kind of people will like a given book; medical clinics can quickly and accurately 
give a diagnosis for a list of symptoms. The basic pattern of converting an input vector 
of data into one item from a set of possible classes is used in applications everywhere. 
Much research is done towards finding more powerful tools to accurately build these classification 
systems.

Evolutionary Computing a powerful and useful tool in many application 
areas in which the search space of potential solutions is very large, and good solutions
can relate to one another. By implementing nature's evolution algorithms in our own ways, we can quickly 
find partial solutions, and combine them into better solutions. This simple process can be 
used in a variety of applications. One such area is artificial neural networks, used to model complex
relationships between inputs and outputs, and discover patterns in data.

This paper utilizes a specific type of evolutionary algorithm, Gene Expression Programming,
and uses it to create artificial neural networks for classification. To do this, a highly expandable 
framework for using Gene Expression Programming is presented, along with some results of tests done 
with it. The paper will begin with a description of the Gene Expression Programming algorithm, and 
how it can be used to create artificial neural networks. This will be followed by an explanation 
of the framework developed, how it works, and some experiments done using the framework. Finally,
the results of the tests will be presented, along with some conclusions, a description of the 
concepts learned, and a short discussion on potential future work with the framework. 


\addcontentsline{toc}{chapter}{Acknowledgements}
\chapter*{Acknowledgements}



\tableofcontents 

\addcontentsline{toc}{chapter}{List of Figures}
\listoffigures

\addcontentsline{toc}{chapter}{List of Tables}
\listoftables

\chapter{Introduction}

\section{Motivation}

Classification and pattern recongition are powerful tools which can be used in a variety of 
fields. Online businesses can best decide what ads to show which users in order to maximize
the likelihood of the ad being noticed and clicked on. Medical systems can help doctors 
decide if a patient is well enough to be sent home. Meteorologists can determine what the 
most likely weather in a given city will be tomorrow. 

All of these tasks require that a given 
training set of data be used to create the most accurate classification system possible. 
This has implications in information theory: how much entropy is there in the data, and 
how high can classification possibly be for a given training set? Research in this field 
always comes down to how accurate the generated classifiers can be. 

The motivation of this project is to create a framework for a specific type of classifier
generation, Gene Expression Programming, to see just how powerful the technique is, and 
allow others to research further using it.

\section{Objectives}

The objectives of the project come down to a set of goals for the final product:

\begin{enumerate}

 \item Build a framework which implements the Gene Expression Programming algorithm.

 \item Make the framework expandable, so that future users can enhance it further, and use it for their purposes. 

 \item Use the framework to see how accurate classifiers generated by it can be. 

\end{enumerate}






\section{Background}
Some background informations.




\chapter{The GEP Framework}


\chapter{Experiments Conducted}



\chapter{Results}


\chapter{Conclusion}


\end{document}          
